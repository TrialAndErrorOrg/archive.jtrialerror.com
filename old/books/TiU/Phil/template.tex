\documentclass[11pt]{article}
\usepackage[utf8]{inputenc}
\usepackage[default]{roboto}
\usepackage{lettrine}
\usepackage[backend=biber, style=apa]{biblatex}
\addbibresource{politicaljusticebib.bib}
\usepackage{sectsty}
\sectionfont{\large\centering}
\date{}

\title{\textbf{POWER, JUSTICE AND IDENTITY\\
    \textit{\Large An Introduction to Political Philosophy}}}

\author{Constanze Binder\footnote{ I am very grateful to Martin van Hees and Lodi Nauta for their detailed editorial comments and very useful suggestions to further improve this chapter. Furthermore, I would like to thank Wiep van Bunge and Han van Ruler for their support in tracing historical sources employed in this chapter.}\\
\href{mailto:binder@esphil.eur.nl}{binder@esphil.eur.nl}}
\usepackage[papersize={170mm,240mm}, lines=42, textwidth=135mm]{geometry}
        \setlength{\parindent}{0pt}
        \setlength{\parskip}{\baselineskip}

\usepackage[colorlinks=true, citecolor=blue, linkcolor=blue, urlcolor=blue, pdftitle={Power, Injustic and Identity}, pdfsubject={An introduction to Political Philosophy}, pdfauthor={Constanze Binder}]{hyperref}
\begin{document}

\maketitle
\section{INTRODUCTION}
\lettrine[lines=3]{I}
magine you could re-design society, say on the regional, national or even global level, without yet knowing which role you would have in it yourself. What would you propose? Which values and rules, if any, should govern society?
Who should rule? What would be the form of government? How can we (if at all) justify the monopoly of power of a state and what government would you choose? How would you distribute economic resources? What (if anything) do we owe future generations and how should this affect our current action, for instance, in the light of climate change or pandemics like corona?
Political philosophy is an area of philosophy and political theory that aims to answer these questions. It draws on an age-old tradition of thinking that focuses on fundamental questions about the organization of human collaboration. An important feature is its normative nature. Political philosophy focuses not so much on how things are but rather on how they ought to be.

At the same time, philosophers are also “children of their time”. That is, their thinking and the questions they raise are influenced by the developments of the world surrounding them, for instance the civil war that inspired Thomas Hobbes to write his Leviathan, the intellectual climate shaped by Rousseau and other Enlightenment thinkers that led to the French Revolution, the Holocaust that shaped Hannah Arendt’s life, or the global inequality and poverty that Martha Nussbaum and Amartya Sen address in their work. This is one of the many fascinating sides of political philosophy: it stands on the shoulders of giants, drawing on the insights of some of the greatest thinkers of (Western) history, but it also allows one to extend and use these insights to shed new light on the burning issues of one’s own time.


Of course, philosophers are also human beings: they have, like all of us, some remarkable blind spots in their thinking, at least so we think from our own vantage point. For Aristotle, for example, the exclusion of slaves and women from the status of democratic citizenship was seemingly so self-evident that it did not seem to require much justification. Similarly, Mary Wollstonecraft, a pioneering 19th century thinker inspiring the movement for women rights to vote, was convinced that the right to vote should be restricted to women of status, leaving them enough time for the required study to engage in politics.


In this light, an engagement with political philosophy is a fascinating and challenging endeavour but vulnerable to the blind spots that characterise our own time and thinking. Developments are going fast and seem often to be unpredictable. Ten years ago, hardly anybody could have predicted the UK leaving the EU or the election of a business man and TV star as president of the United States of America. “Fake news” is new on the political agenda, while other topics have become prominent again, such as the increase in inequality of economic and political power of the more affluent members in our societies.

Why are such tendencies puzzling? Do they refer to developments, as some claim, that negatively affect or even threaten the cornerstones of democracy?
Wouldn’t a government of wise leaders or experts, such as economists or health care specialists, be better suited to manoeuvre society through difficult times of financial crisis or pandemics? Can a society with economic inequality as we now experience it be stable? Can it be just? Can our democracy deal with these developments or do we need to rethink the way we organise our societies to account for the fundamental pre-conditions for a democracy to work?
We start in section 2 with questions about the foundations of our societies and the justification of the state. We then turn, in section 3, to different forms of governments and the reasons to value democracy. Section 4 discusses the notion of justice and how some of the most influential political philosophers of the last century envisioned a just society. Questions about how polarisation in liberal societies may have given rise to populism and the return of nationalist sentiments are the subject of section 5. We conclude in section 6 with an outlook of possible directions that political philosophy might take in the future.

\section{ JUSTIFYING POWER}
\lettrine[lines=3]{O}ne of the core questions with which political philosophers are concerned is: can one justify the existence of the state and, if so, which types of government are to be preferred? Our world is organised in nation states.

Of course, we can read up in history books how the current nation states came into being and how the boundaries on our maps were drawn. However, this does not yet answer the question of how we can justify the existence of states and the transfer of power to those who rule them. What gives the (moral) grounds for the monopoly of power that governments usually possess? What can possibly justify the power to include some and exclude others from admission to a nation state?
Philosophers have often thought about the justification of the state by means of a thought experiment in which one imagines a “state of nature”, that is, a world in which the state does not exist and in which its inhabitants are thus unconstrained by legal rules or external authority. How would such a world look like, how would it be to live in it? Our answers to these questions will affect our assessment of the state. If the state of nature is a situation in which our lives would be miserable, then we may prefer to live in a world where the state does exist.


\textcite{Hobbes1651} famously argued so. He thought the state of nature would be a situation of a war of all against all: without the protection and security offered by the state, rivalry over scarce goods would lead to conflict and ultimately to war.

As result, the individuals in the state of nature will agree that a state is needed and they will, so Hobbes argues, transfer their power to it. According to Hobbes,
the state is justified because it is needed to maintain peace and civil order and because citizens (hypothetically) consent to it.

Is this argument plausible? Hobbes’s pessimism about the prospects of living peacefully in a state-less society was based on a particular view about human beings. \textcite{Locke1689} uses the same thought experiment but is more optimistic about the possibility of human cooperation in the state of nature. Yet, he also concludes that the individuals will decide to form a state and defer some of their powers to it. Such a state possesses a monopoly of force and the protection that it gives to citizens is taken to be superior to the messiness and inconvenience of a state of nature in which individuals themselves try to settle their conflicts.

However, neither Hobbes’s nor Locke’s argument is undisputed. Some anarchists have argued that a peaceful state of nature is possible and that it is to be preferred to the possible abuses of power committed by states. Other defendants of anarchy see the role of the state merely as temporary. The state is necessary until humans have perfected their conduct and are ready for anarchy, that is, the state is a necessary preparation for living without it.

Obviously, the argument from a state of nature is a thought experiment - we have never actually set up or consented to a government in this way. Nevertheless,
the argument is useful in our understanding of the possible justificatory grounds of nation states. If, contrary to the anarchists, one does agree with Hobbes and Locke and takes the thought experiment to give sufficient justification of the state, then the next question is: who should govern us and how should they do so?
\section{WHO SHOULD HAVE THE POWER?}

\lettrine[lines=3]{T}he question of who should govern or rule us can also be put in this way:
Which form of government should we adopt in our society? Should the power be completely in the hands of one person (dictatorship) or a small group (oligarchy) or should it be “a government of the people, by the people,
for the people” (democracy)? I shall limit myself here to a discussion of the justification of democracy.

When inquiring into the meaning and prerequisites of democracy, one of the first difficulties one encounters is that the term is often used in different ways.

Sometimes it is simply a form of window dressing, such as with the former German Democratic Republic (or East Germany) or the Democratic Republic of Congo. But when we consider what we do take to be genuine democracies,
we encounter quite some variety, making it difficult to come up with a definitive definition. To discuss the many models and versions of democracy that have been put forward over the centuries falls of course outside the scope of this chapter.

Instead, we shall look into some of the main reasons as to why democracy is valued. Getting a clearer view on these reasons will provide us with a guide to assess different democratic systems in different contexts and thus possibly also clarifies the demarcation between democracies and other forms of government.

It is helpful to start with an argument that has been raised against democracy:
an examination of this argument sheds light on why we take democracy to be valuable. According to \textcite{Plato-366}, a government of experts - philosopher-
kings as Plato calls them - trained in a detailed and extensive way would be best suited to rule. Only the wise man knows what is best, and the wise man should thus rule. This view presupposes that there is such a thing as a ‘best’ decision

But what makes a political decision good or bad? One answer is to say that a good decision traces “the common good”. A possible objection to this view, i.e. that a wise ruler would be best, is to say that under certain conditions democracies would actually do a “better job” in arriving at the common good than a government of a group of experts, be it philosophers or others. Furthermore,
people may disagree about what the common good is or be uncertain about their interest. In the face of such uncertainty, it may well be that the “wisdom of the crowds”, and thus democracy, is a better way of arriving at the truth than following a singular person or group.

The utilitarians (further discussed in the chapter on ‘moral philosophy’) Jeremy Bentham and John Stuart Mill took the common good to consist of the happiness
(or “utility”) of all people. Though this definition is far from uncontroversial, there are aspects of life that are usually considered to be an undisputed part of the common good, such as people’s livelihood and health, which are better realized in democracies. The economist Amartya Sen has famously argued that famines,
one of the greatest threats to people’s health, have never occurred in a functioning democracy. Similarly, some argue that democracies are more peaceful and less likely to engage in wars than other forms of government such as autocratic rule.

Whether democracies do indeed perform better in these respects, can in the end only be established empirically. But this reveals a potential problem. It suggests that democracy is to be grounded on what philosophers often call ‘instrumental reasons’: we value democracy only as an instrument to achieve other goals such as people’s happiness or stable economic growth.

Are there no reasons why we value democracy for its own sake? Is there an intrinsic value to democracy, that is, a value independent of its effects? A conversation which this author had with a Tunisian activist of the Arab Spring movement brings this question about the intrinsic value of democracy to light:
this activist made clear that while supporting the protests in the beginning, he was now convinced that Tunisians were worse off than before, giving both the economic situation and overall security issues as reasons for a decrease of wellbeing and happiness. He thought therefore that it would have been better to stick with the previous regime rather than to move to democracy. What would you answer him? Suppose you would want to defend the merits of democracy beyond its possible positive impact on our happiness, how would you argue for it?
One prominent intrinsic value attributed to democracy is connected with the importance we ascribe to human freedom and autonomy. More specifically, if we understand freedom in the way of the French philosopher Jean-Jacques Rousseau, namely as “the obedience to a law we give to ourselves”, then irrespective of its outcomes, democracy is to be cherished because it best realizes this freedom. We find it important to have such control (‘autonomy’)
irrespective of whatever further benefits we derive from it.

One can object that autonomy need not be secured in a democracy. We still have to obey laws with which we may disagree. Moreover, it can be the case that the same minorities are structurally outvoted by the majority. What to make of these possibilities? Do they undermine the defence of democracy in terms of its intrinsic value? Not necessarily. Some will argue that the members of minority groups have an equal possibility to influence the law-giving process: to be outvoted need not mean a lack of influence. Others argue that a democracy is not only characterized by majority voting but also by the allocation of civil and political rights. These rights should ensure the absence of abuses of power and allow citizens to pursue their own ideas about how to live their lives as long as it will not interfere with the rights of others to do so as well.

This leads to another aspect of the intrinsic value often attributed to democracy,
namely equality. Irrespective of the outcome of democratic decision-making procedures, all people can participate in them equally. Be it the “one person,
one vote” principle in representative democracy or the equal chance to convince one’s co-citizens in public debates, adult citizens have the same possibility to participate in the democratic procedure. Of course, one might question what such equality requires in order to fully realise it in practice. If better educated people or those with higher incomes have easier access to the political process,
will we still say that equality is realized? Or what about possible unequal chances to cast an informed vote if information is not sufficiently available in the language of a country’s minority group? The answers given to these and similar questions might differ, but they concern what is required to realise democratic equality and underscore its importance.

Even though the exact nature of the intrinsic value that we attach to democracy is thus open for discussion, its importance shows that a defence of democracy need not only refer to its presumed beneficial effects in terms of economic prosperity, peace building or overall happiness. Even if democratic regimes do not improve and in fact would even worsen the economic outlook or the political stability in a country, there are powerful arguments to defend democratic institutions. This, however, does raise the question how to respond to situations in which security or health of the people can be served more effectively by putting democracy, even if only temporarily, on hold. Does a terroristic threat justify a state of emergency that limits civil liberties, as happened in France after the Paris attacks in 2015 for over two years? Similarly, in case of a fundamental health threat as in the Covid-19 pandemic, can states legitimately suspend freedoms, such as the freedom of movement or assembly, and to bypass or suspend regular democratic procedures?

Remaining vigilant and aware that especially in times of fear, be it terrorism or a virus, we may all have a tendency “to rally around the flag” and accept crucial cuts in our civil and democratic liberties that we otherwise would not accept. As Arendt persuasively reminds us, the conditions and dynamics away from democracy are never inevitable. Being aware of the mechanisms that can undermine democracy makes it possible to address and overcome such threats.

In section 5, we will discuss a recent threat to democracy: the declining relevance of facts and truth in politics. But before we go into this, we first analyze what many also take to be another threat: the economic inequality between the world’s richest 1\% and the rest (as prominently identified by French economist Thomas Piketty). Inequalities in economic power and wealth can undermine political equality and democratic processes. This raises the question when the distribution of social and economic goods can be said to be just and when not.

What are the fair terms of cooperation among members of society? In the next section, we will see how this question, which was perhaps the central theme of political philosophy in the last five decades, is addressed.

\section{JUSTICE AND A FAIR DISTRIBUTION OF THE FRUITS OF COOPERATION}
\lettrine[lines=3]{T}hinking of society as a system of cooperation, as discussed in section 2,
raises the question on how the economic and social benefits of a society’s joint endeavours should be distributed. What should be the basis or criteria of justice? How should we distribute our resources? How much inequality is still acceptable, if at all? Should everyone have the opportunity to pursue their idea of what they take to be the good life? If so, what is required for it? Indeed, how should we start reasoning about this question and how should we make sure that personal interests do not bias our judgements about justice?
The philosopher who has made a huge impact on our thinking about these questions is John Rawls, whose A Theory of Justice \parencite{Rawls1971} is one of the most influential philosophical works in the twentieth century. He asks how we can reason about justice and come to an agreement on which institutions characterize a just society. One problem that we often encounter in discussions about this question, is that ideas about justice might be biased by one’s social position, one’s upbringing, status or other characteristics such as gender or religion. How can we then come to an impartial account of justice, that is an account that allows us to say which societal arrangements and institutions are just, irrespective of our position in society (and the effect such arrangements have on us personally)?
Rawls solves this problem with a thought experiment. He postulates the so-
called original position, that is, a hypothetical situation in which members of society discuss which institutional arrangements would be just. The people are behind a “veil of ignorance”, that is, they do not know which position they will have in the future society that will be governed by the principles they are to agree on; neither do they know the class to which they will belong, their gender, religion,
culture or other aspects of their life. Because of this lack of information, people in this original position will think and reason about the rules of society without yet being able to know how they will affect themselves. According to Rawls, this means they necessarily will be impartial when thinking about the principles of justice that should guide their society.

Rawls argues that the parties behind the veil of ignorance would agree on two main principles that institutions in a just society should satisfy:
\begin{enumerate}

\item Each person is to have an equal right to the most extensive total system of basic liberties compatible with a similar system of liberty for all.

\item Social and economic inequalities are to be arranged so that they are both (a)
to the greatest benefit of the least advantaged, and (b) attached to offices and positions open to all under the conditions of fair equality of opportunity. \footnote{\fullcite[pp. 302-303]{Rawls1971}}
%Rawls, John (1971), A Theory of Justice, Cambridge: Harvard University Press; pp. 302-3.
\end{enumerate}
These two principles of justice are ordered lexically, meaning that one first has to safeguard the most extensive set of equal basic liberties for all. The next step is the realization of the second principle, which ensures that all have an equal opportunity and the means necessary to build, revise and pursue what they themselves take to be worthwhile striving for in life, safeguarded by the basic liberties. According to the Rawlsian principles of justice it can thus be just if people in some positions in society earn more, as long as those positions are open to everyone (equal opportunities) and as long as paying them more also benefits the least advantaged people in society. Say if higher earnings for some in society leads to an increase of overall wealth that is used to increase minimal wages.


Next to a lot of praise and admiration, Rawls’s theory of justice has also received quite some criticism. One criticism voiced by his Harvard colleague Robert Nozick is that the Rawlsian principle of distributive justice focuses on the wrong things. To see whether a distribution is just, says Nozick, we should examine how it came about, not whether it fits a certain equality ideal or some other pattern.

To modify a famous example employed by Nozick, consider a society (say societal state A) where resources are distributed according to whatever principle you subscribe to. Say you believe that equality of wealth is of overriding importance and let A be a situation in which this equality is completely realized.

A famous football player, call him Ronaldo, has bargained with his team that every visitor to a match in which he is to play next season, will pay an extra euro for him only. His team mates are happy with the arrangement because it allows them to play with one of the greatest football players and the management is also happy with it because of the success it promises to bring the team. People love to see Ronaldo play, and are enthusiastically paying the extra money to see him. When the season is over, the equal distribution of wealth is no longer existing - Ronaldo ends up much richer than other citizens. If the same scheme repeats for a number of years, inequality will have vastly risen (societal state B).

Why, so Nozick asked, would state B be problematic if this arose by voluntary transactions only? To assess whether a distribution is just, we must, so Nozick argues, not look at its specific features but to the way it came about.

Nozick formulates a principle of justice: if a distribution is just, then any subsequent outcome that results from voluntary actions only also is just. In his Anarchy, State and Utopia, \textcite{Nozick1971} developed this and other principles (e.g. one concerning the justice of the initial situation and another principle about how to deal with violations of justice) that do not rule out widespread inequality. In line with the second (and possibly even the first) principle of justice defended by Rawls, one would support redistributive policies if such redistribution improves the situation of the worst off. But Nozick would empathically oppose such a policy, arguing that such taxation would amount to ``forced labour”.

Another crucial point of criticism of the Rawlsian theory came from a different angle: disagreeing with Nozick and assuming that equality is important, Amartya Sen famously asked: what is it exactly that we find important to equalise for the sake of justice? Different philosophers have given different answers to this question: utilitarians would reply that everybody’s utility needs to be weighted equally, while others focus on, among others, income and wealth. Sen highlights various problems with these answers. Some people’s tastes are very expensive,
and it is unfair to say that we have to make sure that they are just as happy as those who are easily satisfied. With respect to financial means, Sen points out that personal features of a person may greatly affect what people can actually do with it. A disabled person can, for instance, do fewer things with the same income as another person. Personal characteristics such as age, physical condition,
gender, or the social and environmental aspects of one’s surroundings, may affect how one can use one’s economic means. Simply ensuring for everyone to have the same financial means may thus still create huge inequalities between people. It neglects human diversity and the corresponding differences in what one needs to pursue the ends that one values in life.

As a consequence, Sen introduced the notion of capabilities. Capabilities are the freedoms that people have reason to value and that are actually open to them. These freedoms are substantive in the sense that they refer to actual opportunities people have. Thus, the freedom to move around is something that we all have reason to value but it may require different means for a disabled person. Together with Martha Nussbaum, Sen developed this approach into what is now known as the “capability approach”, and which forms a distinct way of thinking about equality and justice.

The question about the ‘currency of justice’ (happiness, income, wealth, and capabilities by no means exhausts the list), has led to a rich debate about distributive justice. Although the political implications of the views of Rawls,
Nozick, Sen and Nussbaum and others on justice differ greatly, and range from the ‘left’ to the ‘right’ on the political spectre, the views are all seen as ‘liberal’
perspectives by taking individual rights, liberty and autonomy as core elements.

The debate has also led to a renewed interest in non-liberal approaches to justice. Marxism, in particular, rejects the individualist perspective. I lack the space to discuss the Marxist alternative here and will instead turn to a more recent critique of liberalism.


\section{IDENTITY, LIBERALISM AND POPULIST POLITICS}
\lettrine[lines=3]{R}awls assumed that the people behind the veil of ignorance do not know the social affiliations and group identities they will have in the society they are designing. Some critics argued that this fails to account for the importance of people’s group membership, which - if ignored - leads to a problematic neglect of structures of domination or exploitation that people qua being members of particularly oppressed groups are subject to. According to these critics, Rawls implicitly projects the values of an intellectual male elite, e.g. such as rationality and individuality, onto the so-called neutral and impartial parties behind the “veil of ignorance”. By doing so he ignores the reality of disadvantaged groups such as women, black people or cultural minorities. Iris Marion \textcite{Young1990}, for instance, argues that this neglect of taking people’s memberships in oppressed or disadvantaged groups into account leads to a continuation of the status quo;
the Rawlsian theory re-enforces the position of the powerful.

To account for the realities of oppressed groups, the role and importance of recognition of different cultures has become more prominent in recent decades.

How should a state with a core commitment to pluralism and state neutrality regarding the objectives that people pursue in life, deal with situations of oppression? Can it justify special requirements of particular groups? How can a framework identify and remedy oppressive structures that disadvantage people qua being members of a group? One answer given by Iris Marion Young herself,
is to shift the focus from the alleged neutrality of Rawls towards a state that acknowledges and cherishes diversity in values and background.


Whereas the importance of group membership and the attention for possibly hidden patterns of oppression can be seen as a critique of the political ‘left’,
the focus on people’s identification with a group or culture (and the values and preferences coming with it) is also seen as major factor of a shift towards the populist politics of Trump and others. Both movements form instances of what is called ‘identity politics’: the appeal to the reality and demands of particular groups, whether these are the identities of oppressed (minority) groups (as in case of the left) or the mobilisation of the more conventional (nationalist) identities
(as often, though not exclusively, exercised by the populist right). In both cases it is a shift that, according to Francis \textcite{Fukuyama2018}, risks to undermine the old ideals of democratic societies by moving away from equality and freedom as core values, towards a more fragmented society in which politicians appeal to voter’s identities.


More generally, what is common to nearly all populist politicians is that they usually claim to speak for or represent “the people”, whose interests they claim to protect against elites or against the effects of migration or (economic)
globalisation. A common interpretation of the unexpected rise in populist politics across the Western world is that an increasing number of people feel left behind by the conventional political parties. This analysis is shared by populists from the left and the right. The exact explanation may subsequently differ. Some refer to economic reasons and express the view that the process of globalization has led to gross inequalities. Others refer to presumed threats to one’s traditional values and cultures, either by migration from “outside” or by new feminist or LGTB movements from “within”. A return to nationalist rhetoric of putting one’s own people first by restricting migration or trade belongs to the standard repertoire of such politicians as does the opposition to all kind of other societal changes.

The increasing polarization within society as a result of identity politics and populism, poses a serious threat to liberal democracies and, as some argue,
to societies in general. One possible response, according to Fukuyama, is to conceive of one’s identity not in narrow terms, but in more inclusive and broader terms. One is not, say a white middle-aged female from the outskirts of Rotterdam, but a Dutch, European or, in the words of Desiderius Erasmus, a citizen of the world.\footnote{\fullcite{Erasmus1522}} To populists, however, who argue for a return to nationalism or patriotism, such a broader ‘cosmopolitan’ perspective is just one of the many unwanted features of globalization.

It is not possible to predict to what extent the different views about the value of cultural diversity and inclusion will continue to dominate the future political agenda. But it is obvious that it is one of the most important challenges to liberalism, both in its ‘left’ and ‘right’ versions, that political philosophy has helped shape the last decades.


\section{CONCLUSION}
\lettrine[lines=3]{O}ur societies are currently confronted with a number of developments such as climate change, pandemics, and migration that transcend the confines of the nation-state. As with populism, the philosophical and political questions which these developments raise, require political philosophers to move beyond the conventional horizons with which they were concerned until the end of the last century. Instead, they will have to broaden their perspectives from nation states to the world at large and from policies here and now towards inter-generational and environmental justice.

Global justice concerns issues of (distributive) justice on a global scale and the duties that states owe towards, for instance, refugees and migrants. Similarly,
questions about inter-generational justice have become particularly prominent because of climate change and its effects. What do we owe our children, and our children’s children and future generations more generally? Are we required to leave a share of the Earth’s resources to them? Do our emission-intensive life styles create duties towards people in the global South, where emissions have started to increase only more recently? These are but some of the timely debates of today’s political philosophers. Inspiration for answers can be found in the work of great thinkers of the past but the new developments and challenges that confront us now also require new answers and new approaches.

\clearpage
\addcontentsline{toc}{section}{RECOMMENDED FURTHER READING}
\section*{RECOMMENDED FURTHER READING}
\fullcite{Kymlicka2001}

\fullcite{Swift2019}

\fullcite{Wolff2016}

\addcontentsline{toc}{section}{BIBLIOGRAPHY}
\section*{BIBLIOGRAPHY}
\nocite{*}
\printbibliography[heading=none]
\end{document}
