% Document class options:
% =======================
%
% serif: Sets the body font to be serif.
%
% twocolumn: Sets the body text in two - column layout.
%
% Fill in the type of article here.
% empirical
  % reflection
  % meta
  % rga
  % editorial
  %
% Using other bibliography styles:
% =======================
% authordate = APA
  % number =[1]
  %
    \documentclass[twocolumn, issue, rga, authordate]{jote-new-article}


  %%% Add the bibliography, make sure it's in the same directory
\addbibresource{dunleavy.bib}

%%% Add additional packages here if required.Usually not needed, except when doing things with figures and tables, god help you then

  % This package is for generating Lorem Ipsum, usage: \lipsum[X] where X is the Xth paragraph o

%%% TODO: Make this a 1 - 5 option scale to reduce the chance of mistyping

  % Enter the title, in Title Case Please
    % Try to keep it under 3 lines
\jotetitle{Are We Meeting Best Practice Standards?: A Longitudinal Analysis of Mental Health Practices Within the Florida Child Welfare System with Implications for Child Well-being}

% List abbreviations here, if any.Please note that it is preferred that abbreviations be defined at the first instance they appear in the text, rather than creating an abbreviations list.
%\abbrevs{
%  ABC, a black cat; DEF, doesn't ever fret; GHI, goes home immediately.}

    % Include full author names and degrees, when required by the journal.
% Use the \authfn to add symbols for additional footnotes and present addresses, if any.Usually start with 1 for notes about author contributions; then continuing with 2 etc if any author has a different present address.
% Format: \author[1, 2]{FirstName LastName} \author[2]{...}
  \author[1]{Daniel Dunleavy}

    % Fill it in again for the PDF metadata.Lame workaround but it works
      % Format: \authorone{...} \authortwo{...}
  \authorone{Daniel Dunleavy}

  \rgainfo{Doris Duke Fellowships for the Promotion of Child Well-Being 2017}
    % List the contribution effort here, they will be listed at the end of the page
      % List the acknowledgments.If there is no companion piece, this is listed below the author info
        %\acknowledgments{Author Two would like to thank Author One for doing all the work while they could slack off.}
% List possible conflict of interest.Will default to saying no conflict exists.
%\interests{Author One was paid for by Big Failed Experiment}
% List funding
    %\funding{}
% Include full affiliation details for all authors
\affil[1]{Center for Translational Behavorial Science, 2010 Levy Ave Building B, Suite B0266, Tallahassee, FL 32310}

    % List the correspondence email of the main correspondent
  \corremail{\href{ daniel.dunleavy@med.fsu.edu } {daniel.dunleavy@med.fsu.edu}}

% Optionally list the present address of one of the authors
    %\presentadd[\authfn{2}]{Department, Institution, City, State or Province, Postal Code, Country}

% Fill in the DOI of the paper

    % Always starts with '10.36850/' and is suffixed with one of the following plus a number
      % e  : empirical
        % r  : reflection
          % mr : meta - research
            % rga: rejected grant application
              % ed : editorial
  \paperdoi{10.36850/rga3}

% Include the name of the author that should appear in the running header
  \runningauthor{Dunleavy}

% The name of the Journal
  \jname{Journal of Trial and Error}

% The year that the article is published
  \jyear{2021}
% The Volume Number
    \jvolume{2}
    \jissue{1}
\setcounter{page}{77}
\jpages{77-84}
% The website that's listed in the bottom right
  \jwebsite{
    https://www.jtrialerror.com}

%%% Only \paperpublished is necessary, any combination of the other two is possible

      % When the paper was received
        % Format: 1 January, 1900
    \paperreceived{16 February, 2021}

% When the paper was accepted
      % Format: 1 January, 1900
    \paperaccepted{17 February, 2021}

% When the paper will be published
      % Format: 1 January, 1900
    \paperpublished{3 March, 2022}
    \paperissued{3 March, 2022}
% When the paper is published but in YYYY - MM - DD format, for the crossmark button
    \paperpublisheddate{2022-03-02}

% The pages of the article, comment out if rolling article
      %\jpages{1 - 12}
% Link to the logo, might be redundant
    

% Fill something here if this is a rolling / online first article, will make ROLLING ARTICLE show up on the first page
  
\usepackage{balance}
%%% Companion Piece

      % Reflection and Empirical articles have each other as companion pieces.Add the DOI, Title, and Abstract of the respective Companion piece here

        %%% Abstract

        % These two set the height and width of the abstract.There's no solution to do this  appear in the abstract, max 7
    \keywordsabstract{
health, best practice guidelines, social work}

%%%%%%%%%%%%%%%%%%%%%%%%%%%%%%%%%%%%%%%%%%%%%%%%%%%
% Document Starts
      %%%%%%%%%%%%%%%%%%%%%%%%%%%%%%%%%%%%%%%%%%%%%%%%%%%

    \begin{document}
%%% This starts the frontmatter, which includes everything that's on the front page execpt the text of the article
    \begin{frontmatter}
    \maketitle
      % Type your abstract between these things.Max 250 words.Be sure to include the \noindent, looks bad otherwise
    \begin{abstract}
Best practice standards are one method by which medical providers ensure
effective care, thus promoting well-being. Though formal guidelines have
been recently implemented to direct and standardize children's mental
healthcare in Florida, little research has evaluated the extent to which
they are executed in practice. This study aims to fill this gap by
analyzing Florida Medicaid data. Individual-level data will be collected
from a 12-month period from a random sample of children, on Medicaid,
with a mental health diagnosis; to: 1) Describe the type and frequency
of mental health services provided to this sample, including to those in
the child welfare system; 2) Evaluate the extent to which Florida's
Psychotherapeutic Medication Treatment Guidelines are adhered to; and 3)
Analyze sociodemographic characteristics, to determine if there are
predictive factors which account for undertreatment/overtreatment. Data
will be coded for congruence with these standards and analyzed using
multinomial logistic regression.
    \end{abstract}
    \end{frontmatter}






\section{Research proposal}



\subsection{General information}



\subsubsection{Institution of employment at time of application}


Florida State University


\subsubsection{Prospective host institution}


Florida State University

\noindent Chapin Hall, University of Chicago


\subsubsection{Main field}


Social Work


\subsubsection{Other fields}


Child Welfare


\subsubsection{Required length}


8/7 pages






\subsection{Description of the proposed research}



\subsubsection{Primary research questions}


\paragraph{Question \#1} \emph{What are the sociodemographic
characteristics of children on Medicaid, diagnosed with a mental health
condition?} Sociodemographic characteristics will further our
understanding of how frequently different groups of children are
diagnosed with a mental health condition. Diagnoses will be
cross-tabulated with age-group, gender, race, placement setting, etc. to
answer related secondary questions regarding diagnostic prevalence by
group.

\paragraph{Question \#2} \emph{What types of treatments are being provided
to children diagnosed with a mental health condition?} Descriptive
statistics about type of mental health treatment (i.e. drug and/or
psychosocial interventions) will further our understanding about how
frequently each type of treatment is provided and how often specific
interventions (e.g. cognitive behavioral therapy {[}CBT{]}, fluoxetine
{[}Prozac{]}) are utilized. This includes identifying trends in
concerning pharmacological practices (e.g. polypharmacy, use of multiple
antipsychotics, off-label antipsychotic use; as outlined in \pdftooltip{{\color{tip}Government
Accountability Office}}{Government Accountability Office. (2012). Concerns Remain about Appropriate Services for Children in Medicaid and Foster Care. http://www.gao.gov/assets/660/650716.pdf},~\citeyear{GovernmentAccountabilityOffice2012}, p. 11). This information can be mapped
across sociodemographic variables to describe what treatments each
subgroup of children (e.g. Medicaid-only vs. child welfare involved
children) is receiving. Knowledge gained from this research question can
identify if particular groups of children are being differentially
treated with respect to their peers in the population; and to determine
how these rates compare with national prescription trends (\pdftooltip{{\color{tip}Olfson,
Marcus, Weissman, \& Jensen}}{Olfson, M., Marcus, S. C., Weissman, M. M., \& Jensen, P. S. (2002). National Trends in the Use of Psychotropic Medications by Children. Journal of the American Academy of Child \& Adolescent Psychiatry, 41(5), 514–521. https://doi.org/10.1097/00004583-200205000-00008},~\citeyear{Olfson2002}) and those within the wider child
welfare system (\pdftooltip{{\color{tip}Raghaven~et al.,}}{Raghavan, R., Zima, B. T., Andersen, R. M., Leibowitz, A. A., Schuster, M. A., \& Landsverk, J. (2005). Psychotropic Medication Use in a National Probability Sample of Children in the Child Welfare System. Journal of Child and Adolescent Psychopharmacology, 15(1), 97–106. https://doi.org/10.1089/cap.2005.15.97} \citeyear{Raghavan2005}).

\vskip10pt


\paragraph{Question \#3} \emph{To what extent are the Florida
Psychotherapeutic Medication Treatment Guidelines being met?} The
``Florida Guidelines'' (\pdftooltip{{\color{tip}Florida~Medicaid Drug Therapy Management Program
for Behavioral Health}}{Florida Medicaid Drug Therapy Management Program for Behavioral Health [FMDRMP]. (2017). Florida Psychotherapeutic Medication Guidelines for Children and Adolescents. http://www.medicaidmentalhealth.org/_assets/file/Guidelines/2016\%20Florida\%20Best\%20Practice\%20Medication\%20Child\%20-\%20Adolescent\%20Guidelines.pdf},~\citeyear{FloridaMedicaidDrugTherapyManagementProgramforBehavioralHealth[FMDRMP]2017}) represent best practice standards. The
guidelines are congruent with the most recent research (\pdftooltip{{\color{tip}American~Academy
of Child and Adolescent Psychiatry}}{American Academy of Child and Adolescent Psychiatry. (2015). Recommendations about the Use of Psychotropic Medications for Children and Adolescents in Child-Serving Systems. https://www.aacap.org/App_Themes/AACAP/docs/clinical_practice_center/systems_of_care/AACAP_Psychotropic_Medication_Recommendations_2015_FINAL.pdf},~\citeyear{AmericanAcademyofChildandAdolescentPsychiatry2015}, p, 29; \pdftooltip{{\color{tip}The~TADS Team}}{Treatment for Adolescents With Depression Study (TADS) Team. (2004). Fluoxetine, Cognitive-Behavioral Therapy, and Their Combination for Adolescents With Depression: Treatment for Adolescents With Depression Study (TADS) Randomized Controlled Trial. JAMA, 292(7), 807. https://doi.org/10.1001/jama.292.7.807},~\citeyear{TADS2004}),
which specifies that all children with complex mental health needs
should be provided psychosocial therapy concurrently with medication,
where possible. Understanding the frequency of guideline adherence will
elucidate how often quality, evidence-based mental health treatments are
provided. While deviation from guidelines may be occasionally justified
(\pdftooltip{{\color{tip}Colbrook,}}{Crimson, M. L., \& Argo, T. (2009). The use of psychotropic medication for children in foster.} \citeyear{Colbrook2005}), their consistent implementation is essential to ensure
that effective treatments are provided, which help protect and promote
child well-being.

\paragraph{Question \#4} \emph{What sociodemographic factors are
predictive of treatment discrepancies?} A multinomial logistic
regression will be conducted to determine if particular sociodemographic
characteristics are predictive of treatment quality. This analysis will
determine whether a child's age-group, gender, race, placement setting,
abuse history, etc. increases the odds that they will receive: 1)
Recommended care (Guideline adherence); 2) Undertreatment (i.e. no
treatment, or treatment lacking medication/psychosocial treatment); or
3) Overtreatment (i.e. polypharmacy, atypical/contraindicated drug use).
This information will allow for greater scrutiny within the field and
ensure that particular groups of children aren't harmed by extreme
deviations in care.


\subsubsection{Research approach}


\paragraph{Data Source} This dissertation will utilize Florida Medicaid
data, which contains records of all children enrolled in Florida's
Medicaid program. This dissertation will also utilize a statewide
administrative database from the Florida Department of Children and
Families (DCF). The \emph{Florida Safe Families Network} {[}FSFN{]}
database contains records of each individual child in Florida's child
welfare system; and will be used to identify a subgroup of children on
Medicaid, in child welfare placement settings, for comparison.

\paragraph{Population} My population of interest is children, on Medicaid,
in Florida, with a mental health diagnosis. A random sample will be
taken from the Florida Medicaid database, from a 12-month period. The
sample drawn will be random, in order to facilitate sound statistical
inferences to the larger population of children (see undefined\pdftooltip{{\color{tip}Berk,}}{Berk, R. A., Western, B., \& Weiss, R. E. (1995). Statistical inference for apparent populations.} \citeyear{Berk2004}, pp.
39-52). A subgroup of this population (i.e. children in the child
welfare system) will be identified within the sample, using their unique
Medicaid ID numbers. Merging the information from these two datasets
will provide a novel data source, thereby increasing the depth of
analyses and the importance of the study's findings.

\paragraph{Data access} Access to the Medicaid and FSFN databases will be
requested in the Spring of 2018, with estimated approval by August of
2018. Permission to use the FSFN database is not an obstacle, as Florida
State University's Institute of Child Welfare (FICW) is housed within
the College of Social Work. FICW is a key partner with the Florida
Department of Children and Families. I've spoken with Dr. Jessica Pryce,
director of FICW and a data analyst at the Florida DCF. Both view access
to FSFN as feasible and straightforward process. In an effort to
investigate the feasibility of using Florida Medicaid data, I am
currently working with an AHCA, Florida-Medicaid data analyst to
coordinate the data request process. Data received from Medicaid and
FSFN will be anonymized, individual-level data. It will be coded into an
SPSS dataset and include, among other variables, information about
age-group, race, gender, placement setting, and treatment types. Other
variables may be collected, pending review of data.

\paragraph{Study design} This longitudinal study aims to assess the extent
to which the Florida Guidelines are adhered to in practice. Data will be
collected from a 12-month time period. All cases meeting
inclusion/exclusion criteria will be assigned a unique ID number. The
sample will be drawn, at random, from this group. Data will be entered
into SPSS. Sociodemographic variables (independent variables) will be
collected and analyzed for each case. Multinomial logistic regression
analyses will be conducted at the end of data collection, to determine
what characteristics are predictive of treatment variables/guideline
adherence (dependent variables).

\paragraph{Limitations} Given the size of Florida Medicaid enrollment,
sample size is not perceived as a limitation. Missing data will be
analyzed, in order to determine if it is Missing Completely at Random
(MCAR). Missing data will be replaced using Multiple Imputation (MI), if
methodologically appropriate. A second, more philosophical limitation
regards psychiatric diagnosis. Given the current state of psychiatric
knowledge, the act of making a diagnosis is a fallible process, relying
on descriptive-level, behavioral criteria (\pdftooltip{{\color{tip}Regier~et al.,}}{Regier, D. A., Narrow, W. E., Clarke, D. E., Kraemer, H. C., Kuramoto, S. J., Kuhl, E. A., \& Kupfer, D. J. (2013). DSM-5 Field Trials in the United States and Canada, Part II: Test-Retest Reliability of Selected Categorical Diagnoses. American Journal of Psychiatry, 170(1), 59–70. https://doi.org/10.1176/appi.ajp.2012.12070999} \citeyear{Regier2013}).
Consequently, some children may be incorrectly diagnosed (false positive
diagnosis), while others not in the population under study are
incorrectly left undiagnosed (false negative diagnosis). As in all
studies of this type, diagnoses are presumed to be correct.


\subsubsection{Analytic methods}


\paragraph{Descriptive statistics} Descriptive statistics (e.g.
frequencies, measures of central tendency, and dispersion) will be
reported for all variables in order to describe the characteristics of
the sample. A cross-tabulation will be conducted to examine the
frequency of different types of treatment services and adherence across
key sociodemographic variables.

\paragraph{Inferential statistics} A multinomial logistic regression was
chosen as the method for calculating the odds that a child will receive
treatment that adheres to the guidelines, or that can be considered a
case of overtreatment or undertreatment. While differences can be noted
at the descriptive level, this analysis will help uncover which groups
are at greatest risk for deviations in care, by use of frequentist
inference methods. The reported odds ratio (OR) serves as the effect
size (\pdftooltip{{\color{tip}Tabachnick~\& Fidell,}}{(N.d.-e). Review of Psychiatry, 17(3), 163–172.} \citeyear{Tabachnick2012}, pp. 464-465) of the relationship
between the predictor (i.e. IVs) and outcome variables (i.e. DVs); and
is reported with the associated confidence interval (CI, 95\%; \pdftooltip{{\color{tip}Cumming
\& Finch}}{Cumming, G., \& Finch, S. (2005). Inference by Eye: Confidence Intervals and How to Read Pictures of Data. American Psychologist, 60(2), 170–180. https://doi.org/10.1037/0003-066x.60.2.170},~\citeyear{Cumming2005}). Confidence intervals provide an estimation of the
precision of the OR (\pdftooltip{{\color{tip}Szumilas,}}{(N.d.-c). Sociological Methodology, 25, 421–458.} \citeyear{Szumilas2010},~p.~227). P-values will be reported,
as a tool for controlling error rates (i.e. the false positive discovery
rate) and will be judged, at a minimum, according to the standard .05
threshold; though recent debates (\pdftooltip{{\color{tip}Benjamin~et al.,}}{Benjamin, D. J., Berger, J. O., Johannesson, M., Nosek, B. A., Wagenmakers, E.-J., Berk, R., Bollen, K. A., Brembs, B., Brown, L., Camerer, C., Cesarini, D., Chambers, C. D., Clyde, M., Cook, T. D., De Boeck, P., Dienes, Z., Dreber, A., Easwaran, K., Efferson, C., … Johnson, V. E. (2018). Redefine Statistical Significance. Nature Human Behaviour, 2(1), 6–10. https://doi.org/10.1038/s41562-017-0189-z} \citeyear{Benjamin2018}; \pdftooltip{{\color{tip}Colquhoun}}{Colquhoun, D. (2014). An Investigation of the False Discovery Rate and the Misinterpretation of p -Values. Royal Society Open Science, 1(3), 140216. https://doi.org/10.1098/rsos.140216},~\citeyear{Colquhoun2014}; \pdftooltip{{\color{tip}Lakens}}{Lakens, D., Adolfi, F. G., Albers, C. J., Anvari, F., Apps, M. A. J., Argamon, S. E., van Assen, M. A. L. M., Baguley, T., Becker, R. B., Benning, S. D., Bradford, D. E., Buchanan, E. M., Caldwell, A., van Calster, B., Carlsson, R., Chen, S.-C., Chung, B., Colling, L. J., Collins, G., … Zwaan, R. A. (2017). Justify Your Alpha [Preprint]. PsyArXiv. https://doi.org/10.31234/osf.io/9s3y6},~\citeyear{Lakens2017}) call for a more nuanced approach. Bayesian
methods, which include the use of credible intervals (\pdftooltip{{\color{tip}Hoekstra,~Morey,
Rouder, \& Wagenmakers}}{Hoekstra, R., Morey, R. D., Rouder, J. N., \& Wagenmakers, E.-J. (2014). Robust Misinterpretation of Confidence Intervals. Psychonomic Bulletin \& Review, 21(5), 1157–1164. https://doi.org/10.3758/s13423-013-0572-3},~\citeyear{Hoekstra2014}), will be considered given the nature of
the dataset (see generally \pdftooltip{{\color{tip}Berk,~Western,~\& Wiess,}}{Berk, R.A. (2004). Regression Analysis: A Constructive Critique. Sage Publications, Inc.} \citeyear{Berk1995}).


\subsubsection{Work plan}



\paragraph{Timeline}


I have completed all coursework required for my degree and have
successfully passed my preliminary examinations. I will begin my
prospectus writing in the beginning of January 2018. I plan on defending
my prospectus in April of 2018 (04/18) Upon completion I can begin
formally securing the data needed for my dissertation. I expect to have
approval and access to the Florida Medicaid database by August of 2018
(08/18). I expect to have approval and access to the Florida Department
of Children and Families' (DCF) Florida Safe Families Network (FSFN)
database by August of 2018. Access to the FSFN database is not viewed as
an obstacle. FSU is home to the Florida Institute for Child Welfare
(FICW), a key partner to Florida's DCF. Data from across a 12-month
period, from these two databases, will be synthesized by September of
2019 (10/19). I expect to have analyzed the data and written my full
dissertation by March of 2020 and defend my dissertation by the end of
April (04/20). I plan on graduating at the end of the Spring semester of
2020 (05/20).


\paragraph{Relationship with your policy practice mentors}


My policy mentor is Mr. Robert ``Bob'' Whitaker. I chose Bob as my
policy mentor because he has strongly influenced my professional career.
My desire to get my PhD and subsequent research has been greatly
informed by his published work and advocacy through the Mad in America
Foundation. I value his depth and breadth of knowledge on psychiatric
medications, his enthusiasm for the issues surrounding mental healthcare
(i.e. diagnosis/etiology, treatment efficacy, long-term treatment
outcomes), and passionate efforts to reform the systems of care.

The deepest areas of my knowledge of mental healthcare are related to
the issues of diagnostic accuracy, treatment effects, and practice
ethics. Bob's work and knowledge complements my own knowledge and work,
in that he has extensively covered issues surrounding long-term
treatment outcomes and effects on well-being, as well as issues
surrounding polypharmacy, off-label prescribing, and best-practice
standards. These latter areas of focus will greatly inform my
dissertation work. Bob's ability to interpret discrepancies in treatment
delivery, will help me to identify and operationalize instances of
undertreatment and overtreatment. Similarly, Bob's knowledge of best
practice standards will help me advocate for practice- and
legislative-level changes in children's mental healthcare.

During the fellowship, I will be regularly in touch with Bob via phone,
video chat, and email. It is also likely that we will meet during the
fellowship timeframe at academic conferences to discuss issues in the
field and progress of my research. I hope to utilize the mentor
relationship to help disseminate my research findings, to reach a
diverse array of mental health stakeholders (e.g. researchers,
clinicians, advocates/policy-makers, consumers/families), and to become
better versed in the nuances of policy-change and policy-advocacy. I am
ecstatic to have the opportunity to learn and grow as a professional and
scholar from Bob.


\subsection{Knowledge utilisation}



\subsubsection{Policy practice implications}


Whatever their inherent limitations, when properly implemented, practice
guidelines help curb extreme deviations in care (i.e. undertreatment,
overtreatment). The implications of this study are twofold. It will: 1)
Improve social and emotional well-being by ensuring that evidence-based
treatments are provided; and to 2) Identify and reduce the harm caused
to children by extreme deviation in care. Identification of groups that
might be at higher risk of receiving undertreatment or overtreatment
will lead to greater institutional oversight and attention by staff in
practice settings. These goals are congruent with the national efforts
to standardize healthcare practice (\pdftooltip{{\color{tip}Institute~of Medicine,}}{of psychotropic medication to foster children. U.C. (n.d.). Davis Journal of Juvenile Law and Policy, 15(2), 247–257.} \citeyear{InstituteofMedicine1990};\pdftooltip{  }{of psychotropic prescriptions. (n.d.). http://www.gao.gov/assets/590/586906.pdf}\citeyear{InstituteofMedicine2011}),
with federal and institutional calls for greater oversight over
childhood medication prescription (\pdftooltip{{\color{tip}Government~Accountability Office}}{Government Accountability Office. (2011). HHS Guidance Could Help Sates Improve Oversight of Psychotropic Prescriptions. http://www.gao.gov/assets/590/586906.pdf; Government Accountability Office. (2012). Concerns Remain about Appropriate Services for Children in Medicaid and Foster Care. http://www.gao.gov/assets/660/650716.pdf; Government Accountability Office. (2017). Foster Care – HHS Has Taken Steps to Support States’ Oversight of Psychotropic Medications, Bust Additional Assistance Could Further Collaboration. http://www.gao.gov/assets/690/681916.pdf},
\citeyear{GovernmentAccountabilityOffice2011, GovernmentAccountabilityOffice2012, GovernmentAccountabilityOffice2017}; \pdftooltip{{\color{tip}Leslie}}{Leslie, L. K., Mackie, T., Dawdon, E. H., Bellonci, C., Schoonover, D. R., Rodday, A. M., \& Hyde, J. (2010). Multi-State Study on Psychotropic Medication Oversight in Foster Care. Tufts Clinical and Translational Science Institute.},~\citeyear{Leslie2010}), and with the concerns of child
welfare researchers (\pdftooltip{{\color{tip}Mackie~et al.,}}{Psychotropic medication use in a national probability sample of children in the child welfare system. (2005). Journal of Child and Adolescent Psychopharmacology, 15(1), 97–106.} \citeyear{Mackie2016}) and legal scholars~(\pdftooltip{{\color{tip}Strawbridge,}}{Whitaker, R. (2010). Anatomy of an epidemic: Magic bullets, psychiatric drugs, and the.} \citeyear{Strawbridge2011}).

\paragraph{Ensuring the provision of evidence-based treatment} Childhood
physical and sexual abuse and neglect are strongly associated with the
development of childhood mental health problems (\pdftooltip{{\color{tip}Administration~for
Children \& Families}}{Administration for children and families. (2012). Information Memorandum: Promoting Social and Emotional Well-Being for Children and Youth Receiving Child Welfare Services. U.S. Department of Health and Human Services. U.S. Department of Health and Human Services. https://www.acf.hhs.gov/sites/default/files/cb/im1204.pdf},~\citeyear{Administrationforchildrenandfamilies2012}) and the development of later adult
psychopathology (\pdftooltip{{\color{tip}Guadiano~\& Zimmerman,}}{Mackie, T. I., Hyde, J., Palinkas, L. A., Niemi, E., \& Leslie, L. K. (2016).} \citeyear{Guadiano2010}; \pdftooltip{{\color{tip}Matheson}}{Regier, D. A., Narrow, W. E., Clarke, D. E., Kraemer, H. C., Kuramoto, S. J., \& Kuhl, E. A. (n.d.).},~\citeyear{Matheson2013};
 \pdftooltip{{\color{tip}Read~\& Bentall}}{Read, J., Os, J., Morrison, A. P., \& Ross, C. A. (2005). Childhood Trauma, Psychosis and Schizophrenia: A Literature Review with Theoretical and Clinical Implications. Acta Psychiatrica Scandinavica, 112, 330–350. https://doi.org/10.1111/j.1600-0447.2005.00634.x},~\citeyear{Read2012}; \pdftooltip{{\color{tip}Read,~van Os, Morrison, \& Pam}}{Read, J., \& Bentall, R. P. (2012). Negative Childhood Experiences and Mental Health: Theoretical, Clinical, and Primary Prevention Implications. The British Journal of Psychiatry, 200, 89–91. https://doi.org/10.1192/bjp.bp.111.096727},~\citeyear{Read2005}). Therefore,
addressing mental health problems of children on Medicaid and in the
child welfare system is essential for promoting current and future
well-being. An analysis of Florida Guideline implementation, will ensure
that children diagnosed with mental health conditions are and will
receive appropriate care.

\paragraph{Preventing the neglect and abuse of children's mental
healthcare.} In Florida, failure to provide proper mental healthcare can
legally be considered a form of child neglect (\pdftooltip{{\color{tip}Child~Welfare Information
Gateway}}{Child Welfare Information Gateway. (2015). Definitions of Child Abuse and Neglect. U.S. Department of Health and Human Services, Children’s Bureau. https://www.childwelfare.gov/pubPDFs/define.pdf},~\citeyear{ChildWelfareInformationGateway2015}). As the Florida Guidelines represent best practice
standards, failure to consistently implement these guidelines in child
welfare systems can be regarded as a form of \emph{institutional} child
neglect. The consequences of neglecting mental health problems are
severe. Untreated depression can lead to a worsening of symptoms,
resulting in suicide, self-harm, decreased academic performance (\pdftooltip{{\color{tip}Kovacs
\& Goldston}}{Kovacs, M., \& Goldston, D. (1991). Cognitive and Social Cognitive Development of Depressed Children and Adolescents. Journal of the American Academy of Child \& Adolescent Psychiatry, 30(3), 388–392. https://doi.org/10.1097/00004583-199105000-00006},~\citeyear{Kovacs1991}), and social deficits ( \pdftooltip{{\color{tip}Geller,~Zimerman,~Williams,
Bolhofner, \& Craney}}{Geller, B., Zimerman, B., Williams, M., Bolhofner, K., \& Craney, J. L. (2001). Bipolar Disorder at Prospective Follow-Up of Adults Who Had Prepubertal Major Depressive Disorder. American Journal of Psychiatry, 158(1), 125–127. https://doi.org/10.1176/appi.ajp.158.1.125},~\citeyear{Geller2001}). In other words, untreated mental health
problems can be detrimental, if not seriously harmful, to a child's
well-being. Likewise, overtreatment (overmedication, polypharmacy, and
off-label prescription) can lead to serious physical and mental health
problems; a trend some call ``pharmacological abuse'' (\pdftooltip{{\color{tip}Healy,}}{Matheson, S. L., Shepherd, A. M., Pinchbeck, R. M., Laurens, K. R., \& Carr, V. J. (2013).} \citeyear{Healy2009}, pp.
272-273). Overtreatment is not uncommon in child welfare systems, where
children are 2-3 times more likely to be medicated than their peers in
the community (\pdftooltip{{\color{tip}Crimson~\& Argo,}}{Florida Medicaid Drug Therapy Management Program for Behavioral Health [FMDRMP]} \citeyear{Crimson2009}; \pdftooltip{{\color{tip}Raghavan}}{Raghavan, R., Zima, B. T., Andersen, R. M., Leibowitz, A. A., Schuster, M. A., \& Landsverk, J. (2005). Psychotropic Medication Use in a National Probability Sample of Children in the Child Welfare System. Journal of Child and Adolescent Psychopharmacology, 15(1), 97–106. https://doi.org/10.1089/cap.2005.15.97},~\citeyear{Raghavan2005}). Foster
care children are estimated to be medicated at a rate 3-13 times that of
their peers (\pdftooltip{{\color{tip}Leslie~et al.,}}{Leslie, L. K., Mackie, T., Dawdon, E. H., Bellonci, C., Schoonover, D. R., Rodday, A. M., \& Hyde, J. (2010). Multi-State Study on Psychotropic Medication Oversight in Foster Care. Tufts Clinical and Translational Science Institute.} \citeyear{Leslie2010}). Children on Medicaid are prescribed
antipsychotics at twice the rate of those privately insured (\pdftooltip{{\color{tip}Government
Accountability Office}}{Government Accountability Office. (2012). Concerns Remain about Appropriate Services for Children in Medicaid and Foster Care. http://www.gao.gov/assets/660/650716.pdf},~\citeyear{GovernmentAccountabilityOffice2012}). This is concerning since overtreatment can
increase the rate of adverse effects (\pdftooltip{{\color{tip}Whitaker,}}{(N.d.-g). National Academy Press.} \citeyear{Whitaker2010}), which include:
akathisia (\pdftooltip{{\color{tip}Loonen~\& Stahl,}}{pictures of data. (n.d.). American Psychologist, 60(2), 170–180.} \citeyear{Loonen2011}), suicidality (\pdftooltip{{\color{tip}Bielefeldt,~Danborg,~\&
Gøtzsche}}{Bielefeldt, A. Ø., Danborg, P. B., \& Gøtzsche, P. C. (2016). Precursors to Suicidality and Violence on Antidepressants: Systematic Review of Trials in Adult Healthy Volunteers. Journal of the Royal Society of Medicine, 109(10), 381–392. https://doi.org/10.1177/0141076816666805},~\citeyear{Bielefeldt2016}; \pdftooltip{{\color{tip}Healy}}{medication oversight for children in foster care: A national examination of states’ monitoring mechanisms. (n.d.). Administration and Policy in Mental Health and Mental Health Services Research, 44(2), 243–257.},~\citeyear{Healy2005}), and physical dependence
syndromes (\pdftooltip{{\color{tip}Fava~et al.}}{Fava, G. A., Gatti, A., Belaise, C., Guidi, J., \& Offidani, E. (2015). Withdrawal Symptoms after Selective Serotonin Reuptake Inhibitor Discontinuation: A Systematic Review. Psychotherapy and Psychosomatics, 84(2), 72–81. https://doi.org/10.1159/000370338},~\citeyear{Fava2015}). Long-term antipsychotic use, is
associated with progressive reduction in brain volume (\pdftooltip{{\color{tip}Fusar-Poli et
al.}}{Fusar-Poli, P., Smieskova, R., Kempton, M. J., Ho, B. C., Andreasen, N. C., \& Borgwardt, S. (2013). Progressive Brain Changes in Schizophrenia Related to Antipsychotic Treatment? A Meta-Analysis of Longitudinal MRI Studies. Neuroscience \& Biobehavioral Reviews, 37(8), 1680–1691. https://doi.org/10.1016/j.neubiorev.2013.06.001},~\citeyear{Fusar-Poli2013}) and Parkinson's-like movement disorders (\pdftooltip{{\color{tip}American~Psychiatric
Association}}{American Psychiatric Association. (1992). Tardive Dyskinesia: A Task Force Report of the American Psychiatric Association. American Psychiatric Association.},~\citeyear{AmericanPsychiatricAssociation1992}). Thus, identifying cases of overtreatment (and
undertreatment) will promote children's physical and mental well-being,
prevent harmful trends in treatment practice, and lead to policy change.


\subsection{Cost estimates}



\subsubsection{Total budget requested}


30.000 \$ US


\subsubsection{Intended starting date}


August 2018


\subsubsection{Application for additional grants}


None


\subsection{Data management plan}


N.A.


\subsection{Ethics}


N.A.


\subsection{Society}



\subsubsection{Public summary}


N.A.


\section{Reviews}


N.A.


\section{Rebuttal}


N.A.


\section{Decision}



\textbf{Doris duke fellowships application}


\noindent Doris Duke Fellowships
\textless DDFellowships@chapinhall.org\textgreater{}

\noindent Tue 3/13/2018 9:18 PM


\noindent To: Daniel Dunleavy \textless djd09e@fsu.edu\textgreater{}
\vskip\baselineskip
\noindent Dear Dan,

Thank you for applying to the \emph{Doris Duke Fellowships for the
Promotion of Child Well-Bein}g and participating in atelephone interview
for finalists. Unfortunately, after careful consideration, the selection
committee did not selectyour application for the fellowship. It was a
very difficult decision that does not necessarily reflect on the quality
orrelevance of your proposed research. In selecting the cohort, every
attempt is made to achieve a \emph{group} of scholarsthat will
complement each other's skills and represent a diversity of disciplinary
perspectives, research interests,and professional experiences. Our final
selection is difficult, particularly when the finalists are as highly
qualifiedas they were this year.

We greatly appreciate your interest in the fellowship and regret that we
are not able to offer you an award. We wishyou the best of luck in
completing your dissertation research and in your future endeavors, and
hope that you willcontinue to focus on the prevention of child abuse and
neglect.
\vskip\baselineskip

\noindent Sincerely,

\noindent Deborah Daro, PhD.

\noindent  Chair of the Selection Committee
\vskip\baselineskip

\noindent \emph{Doris Duke Fellowships for the Promotion of Child Well-Being}

\noindent Senior Research Fellow

\noindent Chapin Hall at the University of Chicago
\vskip\baselineskip

\includegraphics[width=2.23333in,height=0.55833in]{media/image1.jpeg}


\nocite{*}

\printbibliography
\license
\newpage
\section{Appendix}





\subsection{Personal statement}



\subsubsection{Your expected benefits from the fellowship}



\paragraph{Interdisciplinary perspective}


Interdisciplinary work has been an extremely valuable part of my
professional and academic development. In my professional experience, I
have worked as a social worker among physicians, nurses, psychologists,
support-staff, and other social workers. Each has enriched my experience
by offering a unique lens from which to understand my clients and the
challenges faced as part of a professional team. At the academic level,
my education has been unquestionably enriched by taking courses across
fields. It has provided me the opportunity to learn from researchers in
the fields of Sociology, Psychology, Philosophy, Medicine, and Public
Health, and has taught me how to analyze social problems from a variety
of perspectives.

One of the things which I am looking forward to most, as a fellow, is
collaborating with and becoming enriched by the Doris Duke Fellowship
and its multidisciplinary peer learning network. Undoubtedly, the
fellowship's interdisciplinary perspective and emphasis will help shape
my approach and research regarding child well-being, as well as the
scope of the issues surrounding providing mental healthcare to children,
both within and outside the child welfare system. Considering that
previous cohorts have been comprised of fellows from the fields of
Sociology, Psychology, Education, Public Policy, Criminal Justice,
Social Work, and Public Health (among others), I believe that my work
and how I understand these social problems, will only become further
enriched and refined.


\paragraph{Peer learning network}


The feedback that I receive most frequently from my academic peers is
that I have an insatiable appetite for knowledge. I continuously strive
to learn, to understand the world around me, and to solve important
social problems. As a classmate and colleague, I am attentive, offer
constructive feedback, and engage thoughtfully with new ideas, even
where they diverge from my own particular research interests. These are
personal strengths which I can contribute to the work of other fellows.
While I do not pretend to know everything about every topic, I always
try to sincerely and thoughtfully engage with an issue. I try to use my
critical thinking skills to help others refine their ideas and to
provide constructive feedback. I attribute these assets to my
supplementary coursework and interest in philosophy, a field which has
helped me to greatly clarify and explicate my own ideas and arguments.
\balance
As described in the previous section*, one of the things I benefit from
most as a social worker and academic is different perspectives. I
believe the Peer Learning Network will allow me to engage with fellows
from disciplines both similar to and quite different from my own,
exposing me to new ideas, perspectives, arguments, and resources. It is
clear to me that the peer network has great potential to improve and
refine my research and ideas; as well as for me to contribute to the
intellectual development and growth of other fellows. This aspect of the
fellowship is among the most exciting opportunities available to me in
my academic and professional development.





\subsubsection{Looking forward}



\paragraph{Immediate plans following graduation}


\subparagraph{Immediate professional goals} My immediate plans following
graduation are to secure a tenure-track position in a school of Social
Work at a research university in the United States or related policy
center. This will be a place where I can continue researching on child
welfare, mental health, and help foster ``good research practices'' both
within my profession and within child welfare research. Additionally, I
aim to teach classes related to child welfare, psychopathology, and
social policy. While research is my primary motivation for coming into
academia, this latter role of teacher is no less important to me. I
believe that helping develop the next generation of social workers, many
of whom will work with children in some respect, is of the utmost
importance in ensuring that our policies and practices are successfully
implemented.

\subparagraph{Immediate research plans} I hope to extend my dissertation
work, by seeking funding from federal, state, and non-governmental
organizations (NGOs). I aim to analyze mental health practice guidelines
in child welfare systems across the nation, to analyze how children's
mental health services are provided nationally, and to engage with
policy reform to improve child well-being and mental functioning. This
research agenda would allow for the attempted replication of the
findings from my dissertation, a hallmark of rigorous scientific
research.


\paragraph{Long term career aspirations}


My long-term aspirations can be divided into three non-exclusive parts.
My goals are to: 1) Refine our understanding about mental health
treatment effects, particularly with children; to 2) Contribute to the
improvement of children's mental health policy and practice locally and
nationally; and to 3) Improve the quality of child welfare research.

\subparagraph{Goal \#1} My most important long-term goal is to use
information gathered about state guidelines (described in
\emph{Immediate Plans} above), along with knowledge of \emph{long-term}
treatment efficacy/safety (see point 2 below), to ensure that the mental
health guidelines utilized in each state are truly evidence-based and
reflect `best-practice' standards for children. I plan on using
information gathered about discrepancies in guideline adherence to
ensure that these improved standards are effectively implemented with
all vulnerable children, once they are in place.

\subparagraph{Goal \#2} Despite enormous research funding, few studies have
been undertaken describing the \emph{long-term} benefits, harms, and
efficacy of many mental health treatments frequently used with children.
Long-term evaluation will help us to develop more effective policies and
practices for vulnerable children with mental health needs. While
effective guideline implementation will minimize extreme deviations in
care, this goal will evaluate the long-term effects of services.

\subparagraph{Goal \#3} ``Good research'' is founded on rigorous research
methodology, sound statistical inferences, attempted replication and
transparent/open research practices. This is true of child welfare
research and the long-term areas of research outlined above. I aim to
advocate for, research, teach, and practice ``good research practices'',
as exemplified by the \emph{Center for Open Science}, \emph{Society for
Improving Psychological Science}, and others.






\subparagraph{Is there anything else you want to share with the selection committee}


I am thrilled to have the opportunity to be considered as a Doris Duke
Fellow. The Foundation's mission, to develop and disseminate knowledge;
and the Fellowship's goal of creating long-lasting improvement in child
well-being, strongly coheres with my own research goals and ethos. This
is my second time applying for the fellowship. The past year has
provided me the opportunity to clarify and refine my research topics and
goals. I am reapplying because I strongly value the opportunity to
engage with, collaborate, and learn from the Fellowship's members and
multidisciplinary peer learning network. It would greatly improve my
scholarly work and professional development to be a Fellow. I believe my
intellectual drive, critical thinking skills, and strong research
background will make me a valued colleague and collaborator within both
the Doris Duke Foundation and Fellowship program. Thank you to the
Selection Committee for taking the time to consider this application.


\end{document}
